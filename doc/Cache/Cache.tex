\documentclass[a4paper, 11pt]{article}

\usepackage[spanish]{babel}
\usepackage[latin1]{inputenc}
\usepackage[T1]{fontenc}
\usepackage{amsmath}
\usepackage{times}

\addtolength{\voffset}{-3cm}
\addtolength{\textheight}{5cm}
\addtolength{\textwidth}{1cm}
 
\title{Memorias Cache}
\author{Maximiliano A. Eschoyez\\ Programaci�n Eficiente --- UBP}
\date{2010}

\begin{document}

\maketitle

\subsection*{Mediciones}

\begin{itemize}
\item \emph{Hit Rate}: Tasa de cantidad de aciertos en la memoria
  cache, es decir, cuando se va a buscar un dato que est� all�. Como
  este valor suele ser muy alto, se utiliza \emph{Miss Rate}.
\item \emph{Average Memory--Access Time}: Tiempo promedio para acceder
  a un dato:
  \begin{equation*}
    = \text{Hit Time} + \text{Miss Rate} \times \text{Miss Penalty}
       \quad \left[ \eta s \ \text{o clocks} \right]
  \end{equation*}
\item \emph{Miss Penalty}: Tiempo para reemplazar un bloque de memoria
  desde un nivel inferior, incluyendo el reemplazo en la CPU:
  \begin{itemize}
  \item \emph{Access Time}: Retardo desde el nivel inferior
  \item \emph{Transfer Time}: Tiempo de transferencia del bloque
  \end{itemize}
\end{itemize}

\subsection*{Terminolog�a}

\begin{itemize}
\item \emph{Hit}: Los datos se encuentran en alg�n bloque de niveles
  superiores:
  \begin{itemize}
  \item \emph{Hit Rate}: Fracci�n de accesos a memoria encontrados en
    los niveles superiores
  \item \emph{Hit Time}: Tiempo para acceder a los niveles superiores,
    que consiste en tiempo de acceso a la RAM + tiempo para determinar
    \emph{Hit/Miss}
  \end{itemize}
\item \emph{Miss}: Se necesita traer los datos desde niveles
  inferiores:
  \begin{itemize}
  \item \emph{Miss Rate} = 1 - (\emph{Hit Rate})
  \item \emph{Miss Penalty}: el tiempo para reemplazar un bloque en un
    nivel superior + el tiempo para entregar el bloque al procesador
  \end{itemize}
\item \emph{Hit Time} $<<$ \emph{Miss Penalty}
\end{itemize}


\subsection*{Performance}

El tiempo de ejecuci�n de un programa est� dado por
\begin{align}
  T &= I_{\text{count}}  \times \text{CPI} \times T_{\text{cycle}} \\
  I_{\text{count}} &= I_{\text{ALU}} + I_{\text{MEM}} \\
  \text{CPI} &= \frac{I_{\text{ALU}}}{I_{\text{count}}} \times
  \text{CPI}_{\text{ALU}}
  + \frac{I_{\text{MEM}}}{I_{\text{count}}} \times \text{CPI}_{\text{MEM}}
\end{align}

Desarrollando las instrucciones que deben realizarse
\begin{align}
  M_{\text{ALU}} &= \frac{I_{\text{ALU}}}{I_{\text{count}}} \\
  M_{\text{MEM}} &= \frac{I_{\text{MEM}}}{I_{\text{count}}} \\
  1 &= M_{\text{ALU}} + M_{\text{MEM}} \\
  \text{CPI} &= M_{\text{ALU}} \times \text{CPI}_{\text{ALU}}
  + M_{\text{MEM}} \times \text{CPI}_{\text{MEM}} \\
  T &= I_{\text{count}} \times \left( M_{\text{ALU}} \times
    \text{CPI}_{\text{ALU}} + M_{\text{MEM}} \times
    \text{CPI}_{\text{MEM}} \right)
  \times T_{\text{cycle}} \\
  \text{CPI}_{\text{MEM}} &= \text{CPI}_{\text{MEM-HIT}}
  + r_{\text{MISS}} \times \text{CPI}_{\text{MEM-MISS}}
\end{align}

Quedando el tiempo total como
\begin{align}
  T &= I_{\text{count}} \times \left[ M_{\text{ALU}} \times
    \text{CPI}_{\text{ALU}} \right.
  + M_{\text{MEM}} \\
  \ & \ \ \ \ \times \left( \text{CPI}_{\text{MEM-HIT}} +
    r_{\text{MISS}} \left. \times \text{CPI}_{\text{MEM-MISS}} \right)
  \right] \times T_{\text{cycle}} \nonumber
\end{align}

Donde:

\noindent
$T$ = Tiempo total de ejecuci\'on\\
$T_{\text{cycle}}$ = Tiempo de un ciclo de procesador \\
$I_{\text{count}}$ = N\'umero total de instrucciones \\
$I_{\text{ALU}}$ = N�mero de instrucciones de ALU (registro-registro) \\
$I_{\text{MEM}}$ = N�mero de instrucciones de memoria (load/store) \\
$\text{CPI}$ = Promedio de ciclos por instrucci�n \\
$\text{CPI}_{\text{ALU}}$ = Promedio de ciclos por instrucci�n de ALU \\
$\text{CPI}_{\text{MEM}}$ = Promedio de ciclos por instrucci�n de memoria \\
$r_\text{MISS}$ = Tasa de \emph{cache miss} \\
$r_\text{HIT}$ = Tasa de \emph{cache hit} \\
$\text{CPI}_{\text{MEM-MISS}}$ = Promedio de ciclos por \emph{cache miss} \\
$\text{CPI}_{\text{MEM-HIT}}$ = Promedio de ciclos por \emph{cache hit} \\
$M_{\text{ALU}}$ = Mezcla de instrucciones de ALU \\
$M_{\text{MEM}}$ = Mezcla de instrucciones de memoria \\

\subsection*{Ejemplos}

�$T$? = Tiempo total de ejecuci\'on\\
$T_{\text{cycle}} = 0,5\,\eta s $ \\
$I_{\text{count}} = 10^{11}$ \\
$I_{\text{MEM}} = 2 \times 10^{10}$ \\
$\text{CPI}_{\text{ALU}} = 1$ \\
$\text{CPI}_{\text{MEM-MISS}} = 100$ \\
$\text{CPI}_{\text{MEM-HIT}} = 1$ \\

\begin{align*}
  I_{\text{ALU}} &= I_{\text{count}} - I_{\text{MEM}} = 8 \times 10^{10} \\
  M_{\text{ALU}} &= \frac{I_{\text{ALU}}}{I_{\text{count}}} = \frac{8 \times 10^{10}}{10^{11}} = \frac{8}{10} = 0,8 \\
  M_{\text{MEM}} &= \frac{I_{\text{MEM}}}{I_{\text{count}}} = \frac{2 \times 10^{10}}{10^{11}} = \frac{8}{10} = 0,2
\end{align*}

Para $r_{HIT} = 0,9$

\begin{align*}
  r_{MISS} &= 1 - r_{HIT} = 0,1 \\
  \text{CPI}_{\text{MEM}} &= \text{CPI}_{\text{MEM-HIT}} + r_{MISS} \times \text{CPI}_{\text{MEM-MISS}} \\
         &= 1 + \left( 1 - 0,9 \right) \times 100 = 1 + 10 = 11 \\
  T &= 10^{11} \times \left[ \left( 0,8 \times 1 \right) + \left( 0,2 \times 11 \right) \right] \times 5 \times 10^{-10} \\
         &= 150\,s
\end{align*}

Para $r_{HIT} = 0,7$

\begin{align*}
  r_{MISS} &= 1 - r_{HIT} = 0,3 \\
  \text{CPI}_{\text{MEM}} &= \text{CPI}_{\text{MEM-HIT}} + r_{MISS} \times \text{CPI}_{\text{MEM-MISS}} \\
         &= 1 + \left( 1 - 0,7 \right) \times 100 = 1 + 30 = 31 \\
  T &= 10^{11} \times \left[ \left( 0,8 \times 1 \right) + \left( 0,2 \times 31 \right) \right] \times 5 \times 10^{-10} \\
         &= 205\,s
\end{align*}

Para $r_{HIT} = 0,5$

\begin{align*}
  r_{MISS} &= 1 - r_{HIT} = 0,5 \\
  \text{CPI}_{\text{MEM}} &= \text{CPI}_{\text{MEM-HIT}} + r_{MISS} \times \text{CPI}_{\text{MEM-MISS}} \\
         &= 1 + \left( 1 - 0,5 \right) \times 100 = 1 + 50 = 51 \\
  T &= 10^{11} \times \left[ \left( 0,8 \times 1 \right) + \left( 0,2 \times 51 \right) \right] \times 5 \times 10^{-10} \\
         &= 550\,s
\end{align*}

\end{document}

