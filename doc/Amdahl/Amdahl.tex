\documentclass[a4paper, 11pt]{article}

\addtolength{\textwidth}{1cm}

\usepackage[spanish]{babel}
\usepackage[latin1]{inputenc}
\usepackage[T1]{fontenc}
\usepackage{amsmath}

\addtolength{\voffset}{-3cm}
\addtolength{\textheight}{5cm}
 
\title{Ley de Amdhal}
\author{Maximiliano A. Eschoyez}
\date{2010}

\begin{document}

\maketitle

\noindent
$T_0$ = Tiempo de computaci�n sin aceleraci�n \\
$T_A$ = Tiempo de computaci�n con aceleraci�n \\
$T_F$ = Tiempo de la porci�n de computaci�n que puede ser acelerada \\
$g$ = Pico de ganancia de performance para la porci�n de computaci�n acelerada \\
$f$ = Fracci�n de computaci�n no acelerada que puede acelerarse \\
$S$ = Ganancia de velocidad (\emph{speed--up}) de la computaci�n con la aceleraci�n ya aplicada \\

Entonces:
\begin{align*}
  S &= \frac{T_0}{T_A} \\
  f &= \frac{T_F}{T_0} \\
  T_A &= \left(1 - f \right) \times T_0 + \frac{f}{g} \times T_0 \\
  S &= \frac{T_0}{\left(1 - f \right) \times T_0 + \frac{f}{g} \times T_0} \\
\end{align*}
Obteni�ndose
\begin{equation}
  S = \frac{1}{\left(1 - f \right)+ \frac{f}{g}}
\end{equation}

\end{document}