\begin{frame}[fragile]{Probando Software}
    Uso de CPU
    \begin{itemize}
        \item \verb|$ top|
    \end{itemize}
    Tiempo de ejecución
    \begin{itemize}
        \item \verb|$ time exec| $\Rightarrow$ grano grueso
        \item \verb|clock_gettime()| $\Rightarrow$ grano fino
        \begin{lstlisting}[style=consola]
  struct timespec inicio, fin;
  ..
  clock_gettime(CLOCK_REALTIME, &inicio);
  ..
  clock_gettime(CLOCK_REALTIME, &fin);
        \end{lstlisting}
        \item \verb|clock()| $\Rightarrow$ tiempo de CPU
        \begin{lstlisting}[style=consola]
  clock_t tiempo;
  ..
  tiempo = clock();
  printf("CPU %d sec\n", tiempo / CLOCKS_PER_SEC);
        \end{lstlisting}
        % \item \verb|clock()|
        % \item \verb|clock()|
    \end{itemize}
\end{frame}


\begin{frame}[fragile]{\emph{Profiling} con Gprof}
    La herramienta \verb|gprof| permite observar el comportamiento del software
    \begin{itemize}
        \item Debemos compilar con el \emph{flag} \verb|-pg|
        \item La ejecución del programa generará un archivo \verb|gmon.out|
        \item Invocamos a \verb|gprof| para ver los resultados
        \item Si no genera datos \emph{post} ejecución, debemos agregar el \emph{flag} \verb|-no-pie|
        \item Los datos recolectados corresponden a la última ejecución
    \end{itemize}
\end{frame}

\begin{frame}[fragile]{\emph{Profiling} con Gcov}
    La herramienta \verb|gcov| permite observar la cobertura que se realiza sobre código fuente del software
    \begin{itemize}
        \item Debemos compilar con el \emph{flag} \verb|--coverage|, sinónimo de los \emph{flags} \verb|-ftest-coverage| y \verb|-fprofile-arcs|
        \item La ejecución del programa generará dos archivos con extensión \verb|.gcno| y \verb|.gcda|
        \item Invocamos a \verb|gcov| para ver los resultados
        \item Los datos recolectados corresponden al acumulado de las ejecuciones
        \item Al recompilar se reinician los contadores
    \end{itemize}
\end{frame}

